\documentclass{article}
\usepackage{latexsym}
\usepackage{geometry}
\usepackage{amsmath}
\usepackage{amssymb}
\geometry{letterpaper}

%% common definitions
\newcommand{\p}{\partial}
\newcommand{\wt}{\widetilde}
\newcommand{\ov}{\overline}
\newcommand{\mc}[1]{\mathcal{#1}}
\newcommand{\md}{\mathcal{D}}


\newcommand{\GeV}{{\rm GeV}}
\newcommand{\eV}{{\rm eV}}
\newcommand{\Heff}{{\mathcal{H}_{\rm eff}}}
\newcommand{\Leff}{{\mathcal{L}_{\rm eff}}}
\newcommand{\el}{{\rm EM}}
\newcommand{\uflavor}{\mathbf{1}_{\rm flavor}}
\newcommand{\lgr}{\left\lgroup}
\newcommand{\rgr}{\right\rgroup}

\newcommand{\LUV}{\Lambda_{\rm UV}}
\newcommand{\LNC}{\Lambda_{\rm NC}}
\newcommand{\LQCD}{\Lambda_{\rm QCD}}
\newcommand{\Mpl}{M_{\rm Pl}}
\newcommand{\suc}{{{\rm SU}_{\rm C}(3)}}
\newcommand{\sul}{{{\rm SU}_{\rm L}(2)}}
\newcommand{\sutw}{{\rm SU}(2)}
\newcommand{\suth}{{\rm SU}(3)}
\newcommand{\ue}{{\rm U}(1)}
%%%%%%%%%%%%%%%%%%%%%%%%%%%%%%%%%%%%%%%
%  Slash character...
\def\slashed#1{\setbox0=\hbox{$#1$}             % set a box for #1
   \dimen0=\wd0                                 % and get its size
   \setbox1=\hbox{/} \dimen1=\wd1               % get size of /
   \ifdim\dimen0>\dimen1                        % #1 is bigger
      \rlap{\hbox to \dimen0{\hfil/\hfil}}      % so center / in box
      #1                                        % and print #1
   \else                                        % / is bigger
      \rlap{\hbox to \dimen1{\hfil$#1$\hfil}}   % so center #1
      /                                         % and print /
   \fi}                                        %

%%EXAMPLE:  $\slashed{E}$ or $\slashed{E}_{t}$

%%

\begin{document}

\section{Summary of different signs in SU(2) theory}


The sign of $ \lambda^{22a\ SU(N)} $ (i.e. the right-handed superorientational zero-mode in $\mc{N}=2$ ) is different 
w.r.t. eqn. (3.20) in the heterotic paper.
As a consequence, in $\mc{N}=1 $ theory in the small-$\mu$ limit, the signs of $ \lambda_+ $, $ \lambda_- $, $ \psi_+ $
and $ \psi_- $ are different from those in eq.(7.8), (7.9).

As for the supertranslational zero-modes, the $ \mc{N}=2 $ zero-modes agree with (3.18).
In $ \mc{N}=1 $, however, the signs for the $ \mu $-terms in the Dirac equations disagree with (7.2).
As a consequence, the unperturbed profile functions $ \lambda_{s0} $, $ \lambda_{t0} $, $ \psi_{s0} $ and $ \psi_{t0} $ do agree 
with (7.6), whereas $ \lambda_{s1} $, $ \lambda_{t1} $, $ \psi_{s1} $ and $ \psi_{t1} $ {\it dis}agree with (7.7)
in sign.

Summarizing, the following terms require change of sign in the small-$\mu$ limit:
\begin{align*}
%
	& 	\lambda_{s1}~, \qquad\qquad \lambda_{t1}~,  \qquad\qquad \text{not}~\lambda_{s0}~, \qquad\qquad \text{not}~\lambda_{t0}~, \\
%
	& 	\psi_{s1}~, \qquad\qquad \psi_{t1}~,  \qquad\qquad \text{not}~\psi_{s0}~, \qquad\qquad \text{not}~\psi_{t0}~, \\
%
	&	\lambda_+~, \qquad\qquad \lambda_-~, \qquad\qquad \psi_+~, \qquad\qquad \psi_-
\end{align*}

Going to the bifermionic term, which has contributions such as
\[
	\lambda_{t0} \lambda_- ~~+~~ \lambda_{t1} \lambda_+
\]
one finds that this expressions gets modified nontrivially in the small-$\mu$ limit: the first term does require a sign change, while
the second does not.
The same applies to other contribution(s) in the expression for the bifermionic coupling term.



\end{document}
