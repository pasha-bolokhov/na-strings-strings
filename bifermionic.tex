\documentclass{article}
\usepackage{latexsym}
\usepackage{geometry}
\usepackage{amsmath}
\usepackage{amssymb}
\geometry{letterpaper}

%% common definitions
\newcommand{\p}{\partial}
\newcommand{\wt}{\widetilde}
\newcommand{\ov}{\overline}
\newcommand{\mc}[1]{\mathcal{#1}}
\newcommand{\md}{\mathcal{D}}


\newcommand{\GeV}{{\rm GeV}}
\newcommand{\eV}{{\rm eV}}
\newcommand{\Heff}{{\mathcal{H}_{\rm eff}}}
\newcommand{\Leff}{{\mathcal{L}_{\rm eff}}}
\newcommand{\el}{{\rm EM}}
\newcommand{\uflavor}{\mathbf{1}_{\rm flavor}}
\newcommand{\lgr}{\left\lgroup}
\newcommand{\rgr}{\right\rgroup}

\newcommand{\LUV}{\Lambda_{\rm UV}}
\newcommand{\LNC}{\Lambda_{\rm NC}}
\newcommand{\LQCD}{\Lambda_{\rm QCD}}
\newcommand{\Mpl}{M_{\rm Pl}}
\newcommand{\suc}{{{\rm SU}_{\rm C}(3)}}
\newcommand{\sul}{{{\rm SU}_{\rm L}(2)}}
\newcommand{\sutw}{{\rm SU}(2)}
\newcommand{\suth}{{\rm SU}(3)}
\newcommand{\ue}{{\rm U}(1)}
%%%%%%%%%%%%%%%%%%%%%%%%%%%%%%%%%%%%%%%
%  Slash character...
\def\slashed#1{\setbox0=\hbox{$#1$}             % set a box for #1
   \dimen0=\wd0                                 % and get its size
   \setbox1=\hbox{/} \dimen1=\wd1               % get size of /
   \ifdim\dimen0>\dimen1                        % #1 is bigger
      \rlap{\hbox to \dimen0{\hfil/\hfil}}      % so center / in box
      #1                                        % and print #1
   \else                                        % / is bigger
      \rlap{\hbox to \dimen1{\hfil$#1$\hfil}}   % so center #1
      /                                         % and print /
   \fi}                                        %

%%EXAMPLE:  $\slashed{E}$ or $\slashed{E}_{t}$

%%

\begin{document}

In the presence of the orientational modes, the gauge field has non-zero longitudinal
components (hep-th/0403149):
\[
	A^{SU(2)}_\mu ~~=~~ -\,\frac{1}{2}\tau^a\, \epsilon^{abc}S^b \p_k S^c\rho(r) ~,  
		\qquad\qquad  \mu = 0,3
\]
	

The bifermionic coupling with the longitudinal SU(2) gauge field switched on looks as
\begin{align*}
	I_{\zeta\chi} ~~=~~ \int r dr \, & 
			\lgr
			\bigl( \lambda_{t0} \lambda_- ~+~ \lambda_{t1} \lambda_+ \bigr) (\rho(r) - 1) 
			~+~
			\frac{g^2}{2} \bigl( \psi_{t0}\psi_- ~+~ \psi_{t1}\psi_+\bigr)
			(1 - \rho(r)/2)   
			\right.
			\\
%
			& 
			\left.
			~+~
			\frac{g^2}{4} \bigl( \psi_{s0}\psi_- ~-~ \psi_{s1}\psi_+\bigr)\,\rho(r)
			\rgr
\end{align*}
The relative sign in the last term seems strange, but I checked that it is such

In the limit small $\mu$, also with all signs taken into account, the expression turns into
\[
	I_{\zeta\chi} ~~=~~ -\, \frac{\mu g^2}{2\sqrt{2}} \,
			\int r\, dr \,
		\lgr  \frac{ g^2 (\phi_1^2 ~-~ \phi_2^2)^2} {\phi_2^2} 
			    \left( 1 ~+~ \frac{1}{2} f ~+~ \frac{3}{2} f_3 \right) ~~+~~
			4\, g^2 (\phi_1^2 ~-~ \phi_2^2 ) f_3 \rgr ~.
\]
The latter term in the bracket is integrable.
This expression agrees with that in the SU(N) case.

%In the limit of small $\mu$, with $ \rho = 1 - \phi_1/\phi_2 $ substituted, this expression turns into

(The old expression was:
\[
	I_{\zeta\chi} ~~=~~ -\,\frac{\mu g^2}{2\sqrt{2}} \, 
			\int r\, dr \, 
			\frac{g^2 (\phi_1^2 - \phi_2^2)^2}{\phi_2^2}
			\lgr (f + f_3) ~-~ \frac{f - f_3}{2} ~-~ 1 \rgr~
\]
\qquad\qquad\qquad\qquad\qquad\qquad\qquad\qquad\qquad\qquad\qquad\qquad\qquad\qquad\qquad\qquad\qquad\qquad)

\end{document}
